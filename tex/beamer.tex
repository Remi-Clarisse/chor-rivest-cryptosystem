\documentclass{beamer}


\usepackage[utf8]{inputenc}
\usepackage[T1]{fontenc}
\usepackage[francais]{babel}
\usepackage{amsmath, amsthm, amssymb, mathabx}
\usepackage{tikz}
\usepackage{algorithm, algorithmic}
\usepackage{array, booktabs}
\usepackage{cite}
\usepackage{tikz}
\usepackage{hyperref}

\newtheorem{theo}{Théorème}[section]
\newtheorem{lemm}[theo]{Lemme}
\newtheorem{prop}[theo]{Proposition}
\newtheorem{coro}[theo]{Corollaire}
\theoremstyle{definition}
\newtheorem{defi}[theo]{Déinition}
\theoremstyle{remark}
\newtheorem{rema}[theo]{Remarque}
\newtheorem{exem}[theo]{Exemple}
\newtheorem{appl}[theo]{Application}
\newtheorem{heur}[theo]{Heuristique}


\def\N{\mathbb N}
\def\A{\mathbb A}
\def\Z{\mathbb Z}
\def\Q{\mathbb Q}
\def\R{\mathbb R}
\def\C{\mathbb C}
\def\K{\mathbb K}
\def\F{\mathbb F}
\def\O{O}
\def\o{o}
\def\gf #1{\mathbb{F}_{#1}}
\def\frob{\operatorname{Frob}}
\def\card{\operatorname{Card}}
\def\car{\operatorname{car}}
\def\pgcd{\operatorname{pgcd}}
\def\ppcm{\operatorname{ppcm}}
\def\id{\operatorname{id}}
\def\aut{\operatorname{Aut}}
\def\hom{\operatorname{Hom}}
\def\isom{\operatorname{Isom}}
\def\gal{\operatorname{Gal}}
\def\mbf #1{\mathbf{#1}}
\def\NP{\mathbb{NP}}
\def\gen #1{\left\langle#1\right\rangle}
\def\ceil #1{\left\lceil#1\right\rceil}
\def\floor #1{\left\lfloor#1\right\rfloor}

\newcommand{\extension}[2]{{#1} / {#2}} % #1 grand corps et #2 petit corps

\floatname{algorithm}{Algorithme}
\renewcommand{\algorithmicrequire}{\textbf{Entrée :}}
\renewcommand{\algorithmicensure}{\textbf{Sortie :}}
\renewcommand{\algorithmicend}{\textbf{fin}}
\renewcommand{\algorithmicif}{\textbf{si}}
\renewcommand{\algorithmicthen}{\textbf{alors}}
\renewcommand{\algorithmicelse}{\textbf{sinon}}
\renewcommand{\algorithmicfor}{\textbf{pour}}
\renewcommand{\algorithmicforall}{\textbf{pour tout}}
\renewcommand{\algorithmicdo}{\textbf{faire}}
\renewcommand{\algorithmicwhile}{\textbf{tant que}}
\renewcommand{\algorithmicloop}{\textbf{boucle}}
\renewcommand{\algorithmicrepeat}{\textbf{repéter}}
\renewcommand{\algorithmicuntil}{\textbf{jusqu'à}}
\renewcommand{\algorithmicprint}{\textbf{afficher}}
\renewcommand{\algorithmicreturn}{\textbf{retourner}}
\renewcommand{\algorithmictrue}{\textbf{vrai}}
\renewcommand{\algorithmicfalse}{\textbf{faux}}
\usetheme{Warsaw}

\title{Étude du cryptosystème de Chor-Rivest}
\author{Rémi {Clarisse} \\ Tuteurs: Daniel {Augot} et Luca {De Feo}}
\institute{ INRIA Saclay--Île-de-France \\ Université de Bordeaux}
\date{Mercredi 13 Septembre 2017}
\subject{Oral de Stage}

\beamertemplatenavigationsymbolsempty
\begin{document}

\begin{frame}[plain]
	\titlepage
\end{frame}

\begin{frame}
	\frametitle{Sommaire}
  	\tableofcontents
\end{frame} 

\section{Le problème du calcul de logarithme discret}
\subsection{La recherche exhaustive}
\begin{frame}
	\frametitle{}
  	
\end{frame} 

\subsection{L'algorithme de Pohlig-Hellman}
\begin{frame}
	\frametitle{}
  	
\end{frame} 

\subsection{L'algorithme des pas de bébé -- pas de géant}
\begin{frame}
	\frametitle{}
  	
\end{frame} 

\subsection{L'algorithme rho de Pollard}
\begin{frame}
	\frametitle{}
  	
\end{frame}

\section{Le cryptosystème de Chor-Rivest}
\begin{frame}
	\frametitle{}
  	
\end{frame} 

\section{La cryptanalyse de Serge Vaudenay}
\subsection{L'équivalence entre les clés privées}
\begin{frame}
	\frametitle{}
  	
\end{frame} 

\subsection{Les étapes de la cryptanalyse}
\begin{frame}
	\frametitle{}
  	
\end{frame} 

\section{Le calcul de logarithme discret selon Antoine Joux}
\subsection{La méthode de calcul d'indice}
\begin{frame}
	\frametitle{}
  	
\end{frame} 

\subsection{Les idées d'Antoine Joux}
\begin{frame}
	\frametitle{}
  	
\end{frame}

\end{document}
